\documentclass[11pt,compress,xcolor=table]{beamer}

\usepackage[portuges]{babel}
\usepackage{lmodern}
\usepackage[utf8]{inputenc}
\usepackage{amssymb,amsmath}
\usepackage[T1]{fontenc}
\usepackage{textcomp}
\usepackage{verbatim}
\usepackage{bold-extra}

% pacote para colorir tabelas
\usepackage{color}
\usepackage{colortbl}
\usepackage{xcolor}
\usepackage[table]{xcolor}

%Colorir arquivos com códigos fontes fontes
\usepackage{minted}
%caixas de textos
%\usepackage{fancybox}

%===== Pacotes para colorir links =====
\usepackage{fancyhdr}
%\usepackage[colorlinks,linkcolor=blue,hyperindex]{hyperref}
%\hypersetup{backref,  pdfpagemode=FullScreen, colorlinks=true,linkcolor=blue}

%===== Pacotes para mostrar códigos fontes =====
\usepackage{listings}
% simpsons
%\usepackage{simpsons}

% Caminhos das Imagens
\graphicspath{{images/}}

% definição de comandos
  \newcommand{\degree}{\ensuremath{^\circ}}
\makeatletter
  \newcommand\tinyv{\@setfontsize\tinyv{6pt}{6pt}}
\makeatother

%===== Configurações para mostrar Códigos Fonte ===== %
\lstset{numbers=left,
  language=python,
  stepnumber=1,
  firstnumber=1,
  numberstyle=\tiny,
  extendedchars=false,
  escapeinside='',
  breaklines=true,
  frame=tb,
  basicstyle=\tiny,
  stringstyle=\ttfamily,
  showstringspaces=false
  backgroundcolor=\icolor{gray}
  morecomment=[l]{//} % displays comments in italics (language dependent)
}

% Tema da Apresentação
\usetheme{Ilmenau}
\setbeamercovered{transparent}
\setbeamertemplate{footline}[frame number]


% Informações sobre a apresentação
\author[Sobrenome]{Nome}
\title[Título menor]{Título maior}
\subtitle[Título menor]{Título maior}
\institute[abriviação]{completa}
\date[abriviada]{completa}

\begin{document}

\begin{frame}
  \titlepage
\end{frame}

%-------------------------------------------------------------
%-------------------------------------------------------------

%================ Slide Sumario===============================
% cria o sumário
\begin{frame}
\frametitle{Sumário}
\tableofcontents[pausesections]
\end{frame}
%------------------------------------------------------------
%============================================================

\section{Introdução}

\subsection{Latex scripts}
%----------------------------------------------------------------------------%

\begin{frame}
   \begin{block}{}
      ...
   \end{block}
\end{frame}

\begin{frame}
  \frametitle{Resolução}
  \inputminted[linenos,fontsize=\scriptsize]{c}{codes/01-valor-absoluto.c}
\end{frame}

%----------------------------------------------------------------------------%

%\bibliographystyle{sbc}
\bibliography{bibliography/_sbc-template}


\end{document}


